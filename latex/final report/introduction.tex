\section{Introduction and motivation}
Contemporary so-called second-generation gravitational-wave detectors, such as Advanced LIGO \cite{AdvLIGOsite, Harry2010},
 Advanced VIRGO \cite{AdvVIRGOsite}, and LCGT \cite{LCGTsite}, which are under construction now, will be quantum noise limited.
Two quantum noises will limit the sensitivity: the phase fluctuations of the light inside the interferometer (shot noise) at high gravitational wave frequencies
and the random force created by the amplitude fluctuations of the light (radiation-pressure noise) at low frequencies.
For balanced detector the best sensitivity point - where these two noise sources become equal -
 is known as the Standard Quantum Limit (SQL) \cite{92BookBrKh}.
In the linear position meter (the gravitational-wave interferometer is special case of it)
the shot noise corresponds to the measurement noise and radiation pressure noise - to the back-action noise.
 Spectral densities of these noises obey the Heisenberg uncertainty relation \cite{92BookBrKh}.
The SQL is not an absolute limit for measurement precision: there are few methods of overcoming it in gravitational-wave detectors. The most well-known examples are
Quantum Non-Demolition (QND) measurements and Back-Action Evading (BAE) measurements (see, {\it e.g.},
 \cite{90a1BrKh, Buonanno2001m, 01a2Kh, 02a1KiLeMaThVy}). The first method supposes using the Hamiltonian
 of interaction the test body and measurement device,
which commutes with operator of measured quantity \cite{77a1eBrKhVo, 90a1BrKh, 96a1BrKh, 92BookBrKh}. The second method uses the
correlation between the measurement noise and the back-action noise.\cite{Unruh1982, 87a1eKh, JaekelReynaud1990, Pace1993, 96a2eVyMa, 02a1KiLeMaThVy}

In planned third generation detectors,like Einstein Telescope gravitational-wave detector \cite{Sathyaprakash2011} using these methods is supposed,
 that should provide sensitivity better than the SQL. However, both of them require sufficient
 modification of existing experimental schemes and have some technical difficulties (for example, variational-readout scheme is sensitive to optical losses [?]).

In our work we want to investigate the other approach - adaptive linear measurements. The idea is to change
 the parameters of the scheme depending on the results of previous measurements. Our aim was to create the effective algorithm for quantum adaptive measurements,
 check whether we can overcome SQL and compare it to the existing methods.


